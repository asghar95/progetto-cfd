%%%%%%%%%%%%%%%%%%%%%%%%%%%%%%%%%%%%%%%%%%%%%%%
%%% Template for lab reports used at STIMA
%%%%%%%%%%%%%%%%%%%%%%%%%%%%%%%%%%%%%%%%%%%%%%%

%%%%%%%%%%%%%%%%%%%%%%%%%%%%%% Sets the document class for the document
% Openany is added to remove the book style of starting every new chapter on an odd page (not needed for reports)
\documentclass[11pt,english, openany]{book}

%%%%%%%%%%%%%%%%%%%%%%%%%%%%%% Loading packages that alter the style
\usepackage[]{graphicx}
\usepackage[]{color}
\usepackage{alltt}
\usepackage[T1]{fontenc}
\usepackage[utf8]{inputenc}
\setcounter{secnumdepth}{3}
\setcounter{tocdepth}{3}
\setlength{\parskip}{\smallskipamount}
\setlength{\parindent}{0pt}

% Set page margins
\usepackage[top=100pt,bottom=100pt,left=68pt,right=66pt]{geometry}

% Package used for placeholder text
\usepackage{lipsum}

% Prevents LaTeX from filling out a page to the bottom
\raggedbottom

% Adding both languages
\usepackage[english]{babel}

% All page numbers positioned at the bottom of the page
\usepackage{fancyhdr}
\fancyhf{} % clear all header and footers
\fancyfoot[C]{\thepage}
\renewcommand{\headrulewidth}{0pt} % remove the header rule
\pagestyle{fancy}

% Changes the style of chapter headings
\usepackage{titlesec}
\titleformat{\chapter}
   {\normalfont\LARGE\bfseries}{\thechapter.}{1em}{}
% Change distance between chapter header and text
\titlespacing{\chapter}{0pt}{50pt}{2\baselineskip}

% Adds table captions above the table per default
\usepackage{float}
\floatstyle{plaintop}
\restylefloat{table}

% Adds space between caption and table
\usepackage[tableposition=top]{caption}

% Adds hyperlinks to references and ToC
\usepackage{hyperref}
\hypersetup{hidelinks,linkcolor = black} % Changes the link color to black and hides the hideous red border that usually is created

% If multiple images are to be added, a folder (path) with all the images can be added here 
\graphicspath{ {Figures/} }

% Separates the first part of the report/thesis in Roman numerals
\frontmatter


%%%%%%%%%%%%%%%%%%%%%%%%%%%%%% Starts the document
\begin{document}

%%% Selects the language to be used for the first couple of pages
\selectlanguage{english}

%%%%% Adds the title page
\begin{titlepage}
	\clearpage\thispagestyle{empty}
	\centering
	\vspace{1cm}

	% Titles
	% Information about the University
	{\normalsize Computational Fluid Dynamics \\ 
		Dipartimento di Scienze e Tecnologie Aerospaziali \\
		Politecnico di Milano \par}
		\vspace{3cm}
	{\Huge \textbf{NUMERICAL SIMULATIONS OF\\ \vspace{0.1cm} AIRFOILS IN TANDEM\\ \vspace{0.1cm} CONFIGURATION WITH FLAP ON\\ \vspace{0.4cm} LEADING AIRFOIL}} \\
	%\vspace{1cm}
	%{\large \textbf{xxxxx} \par}
	\vspace{4cm}
	{\normalsize ARCURI ROSARIO 945148\\ % \\ specifies a new line
	             ASGHAR ALI RAZA 920035\\
	             CICOLINI GIANMARCO 905732\par}
	\vspace{1cm}
    
    \centering \includegraphics[scale=0.4]{logo1.pdf}
    
    \vspace{0.5cm}
		
	% Set the date
	{\normalsize 11-11-2019 \par}
	
	\pagebreak

\end{titlepage}

% Adds a table of contents
\tableofcontents{}

%%%%%%%%%%%%%%%%%%%%%%%%%%%%%%%%%%%%%%%%%%%%%%%%%%%%%%%%%%%%%%%%%%%%%%%%%%%%%%%%%%%%%%%%%%%%
%%%%%%%%%%%%%%%%%%%%%%%%%%%%%%%%%%%%%%%%%%%%%%%%%%%%%%%%%%%%%%%%%%%%%%%%%%%%%%%%%%%%%%%%%%%%
%%%%% Text body starts here!
\mainmatter

\chapter{Summary}\label{chapt:sum}

This report presents numerical simulations of two airfoils in tandem configuration with a flap on the leading airfoil. \\
The study begins with the applications of the numerical methods for solving Euler and complete Navier-Stokes equations on 4-digits naca airfoils for which the solution (in terms of pressure, entropy and velocity distribution) is well known in order to verify the reliability of the code.\\
Once that task is completed, a flap is added to the main lifting surface (leading surface), a naca 4416, and the results obtained are compared to the data obtained by nasa from experiments. \footnote{The article from which experimental data are taken https://ntrs.nasa.gov/archive/nasa/casi.ntrs.nasa.gov/19740013521.pdf}

Finally a (smaller) trailing surface is added and a parametric analysis is performed varying the (horizontal and vertical) distance between the surfaces, the angle of attack and the flap deflection angle.

The goal of the work is to compare lift and drag of the isolated airfoils with the values obtained in tandem configuration hence to find the effects of the primary lifting surface (upstream) on the secondary (downstream). \footnote{This can be considered a 2D approximation of the flow on wing and tail surfaces of a plane}

\chapter{Problem definition and background}
[\textit{Introduce the problem and state the objective of your work. Briefly present the state of the art regarding the chosen topic and report a reference solution (i.e. numerical or experimental, or the exact one if available).}]
\section{Literature review}
\section{Reference solution}

\chapter{Design of Experiment}\label{chapt:doe}
[\textit{Describe the process used to meet the project goal.}]

\chapter{Computational model}\label{chapt:model}
[\textit{Describe thoroughly the computational model/s used in the project}]
\section{Problem geometry and setup}
\section{Mesh generation and description}
\section{Numerical schemes}

\chapter{Results}\label{chapt:results}
[\textit{ Report the results of the simulations. Validate your work, i.e. show that the computational model (\ref{chapt:model}) and the simulations you run (the DoE \ref{chapt:doe}) were able to obtain the goal of the project}]
\section{Test 1}
\subsection{Grid convergence}
% \section{Test 2} ... as needed

\chapter{Conclusions}

\pagebreak


% Adding a bibliography if citations are used in the report
\bibliographystyle{plain}
\bibliography{bibliography.bib}
% Adds reference to the Bibliography in the ToC
\addcontentsline{toc}{chapter}{\bibname}

\pagebreak

\chapter*{Appendix A: Resources}
[\textit{Report the config files of the software used (i.e. SU2 \cite{economon2015su2} and the mesher). Also attach to this report an archive with the mesh files, solutions and the reference solution data (e.g. data points of a Cp plot ...)}]
\section*{Mesh configuration files}
\section*{SU2 configuration files}
% \section{Reference solution data}


\end{document}
